%-------------------------------------
% LaTeX CV
% Author : Yuan Xue (Soso)
% License : MIT
%-------------------------------------

\documentclass[letterpaper,12pt]{article}[leftmargin=*]

\usepackage[empty]{fullpage}
\usepackage{enumitem}
\usepackage{ifxetex}
\ifxetex
  \usepackage{fontspec}
  \usepackage[xetex]{hyperref}
\else
  \usepackage[utf8]{inputenc}
  \usepackage[T1]{fontenc}
  \usepackage[pdftex]{hyperref}
\fi
\usepackage{fontawesome}
\usepackage[sfdefault,light]{FiraSans}
\usepackage{anyfontsize}
\usepackage{xcolor}
\usepackage{tabularx}

%-------------------------------------------------- SETTINGS HERE --------------------------------------------------
% Header settings
\def \fullname {Yuan Xue (Soso)}
\def \subtitle {Single-cell data science, biochemistry, machine learning, \faLinux}
\def \subsubtitle {Bioengineering Ph.D. Candidate @ Quake lab, Stanford}

\def \linkedinicon {\faLinkedin}
\def \linkedinlink {https://linkedin.com/in/yuan-xue-8b1b254a/}
\def \linkedintext {linkedin/yuan-xue}

% \def \phoneicon {\faPhone}
% \def \phonetext {+1}

\def \emailicon {\faEnvelope}
\def \emaillink {mailto:yuanxue@stanford.edu}
\def \emailtext {yuanxue@stanford.edu}

\def \githubicon {\faGithub}
\def \githublink {https://github.com/xuesoso}
\def \githubtext {github/xuesoso}

\def \websiteicon {\faGlobe}
\def \websitelink {https://xuesoso.github.io}
\def \websitetext {xuesoso.github.io}

\def \headertype {\singlecol} % \singlecol or \doublecol

% Misc settings
\def \entryspacing {-0pt}

\def \bulletstyle {\faAngleRight}

% Define colours
\definecolor{primary}{HTML}{000000}
\definecolor{secondary}{HTML}{8C1515}
\definecolor{accent}{HTML}{263238}
\definecolor{links}{HTML}{1565C0}

%------------------------------------------------------------------------------------------------------------------- 

% Defines to make listing easier
\def \linkedin {\linkedinicon \hspace{3pt}\href{\linkedinlink}{\linkedintext}}
% \def \phone {\phoneicon \hspace{3pt}{ \phonetext}}
\def \email {\emailicon \hspace{3pt}\href{\emaillink}{\emailtext}}
\def \github {\githubicon \hspace{3pt}\href{\githublink}{\githubtext}}
\def \website {\websiteicon \hspace{3pt}\href{\websitelink}{\websitetext}}

% Adjust margins
\addtolength{\oddsidemargin}{-0.55in}
\addtolength{\evensidemargin}{-0.55in}
\addtolength{\textwidth}{1.1in}
\addtolength{\topmargin}{-0.6in}
\addtolength{\textheight}{1.1in}

% Define the link colours
\hypersetup{
    colorlinks=true,
    urlcolor=links,
}

% Set the margin alignment 
\raggedbottom
\raggedright
\setlength{\tabcolsep}{0in}

%-------------------------
% Custom commands

% Sections
\renewcommand{\section}[2]{\vspace{5pt}
  \colorbox{secondary}{\color{white}\raggedbottom\normalsize\textbf{{#1}{\hspace{7pt}#2}}}
}

% Entry start and end, for spacing
\newcommand{\resumeEntryStart}{\begin{itemize}[leftmargin=2.5mm]}
\newcommand{\resumeEntryEnd}{\end{itemize}\vspace{\entryspacing}}

% Itemized list for the bullet points under an entry, if necessary
\newcommand{\resumeItemListStart}{\begin{itemize}[leftmargin=4.5mm]}
\newcommand{\resumeItemListEnd}{\end{itemize}}

% Resume item
\renewcommand{\labelitemii}{\bulletstyle}
\newcommand{\resumeItem}[1]{
  \item\small{
    {#1 \vspace{-2pt}}
  }
}

% Entry with title, subheading, date(s), and location
\newcommand{\resumeEntryTSDL}[4]{
  \vspace{-1pt}\item[]
    \begin{tabularx}{0.97\textwidth}{X@{\hspace{40pt}}r}
      \mbox{\textbf{\color{primary}#1}} & {\firabook\color{accent}\small#2} \\
      \textit{\color{accent}\small#3} & \textit{\color{accent}\small#4} \\
    \end{tabularx}\vspace{-8pt}
}

% Entry with title and date(s)
\newcommand{\resumeEntryTD}[2]{
  \vspace{-1pt}\item[]
    \begin{tabularx}{0.97\textwidth}{X@{\hspace{40pt}}r}
      \textbf{\color{primary}#1} & {\firabook\color{accent}\small#2} \\
    \end{tabularx}\vspace{-6pt}
}

% Entry for special (skills)
\newcommand{\resumeEntryS}[2]{
  \item[]\small{
    \textbf{\color{primary}#1 }{ #2 \vspace{-6pt}}
  }
}

% Double column header
\newcommand{\doublecol}[6]{
  \begin{tabularx}{\textwidth}{Xr}
    {
      \begin{tabular}[c]{l}
        \fontsize{35}{45}\selectfont{\color{primary}{{\textbf{\fullname}}}} \\
        {\textbf{\subtitle}} \\ {\subsubtitle} % You could add a subtitle here
      \end{tabular}
    } & {
      \begin{tabular}[c]{l@{\hspace{1.5em}}l}
        {\small#4} & {\small#1} \\
        {\small#5} & {\small#2} \\
        {\small#6} & {\small#3}
      \end{tabular}
    }
  \end{tabularx}
}

% Single column header
\newcommand{\singlecol}[6]{
  \begin{tabularx}{\textwidth}{Xr}
    {
      \begin{tabular}[b]{l}
        \fontsize{35}{45}\selectfont{\color{primary}{{\textbf{\fullname}}}} \\
          {\textbf{\subtitle}} \\ {\subsubtitle} % You could add a subtitle here
      \end{tabular}
    } & {
      \begin{tabular}[c]{l}
        {\small#1} \\
        {\small#2} \\
        {\small#3} \\
        {\small#4} \\
        {\small#5} \\
        {\small#6}
      \end{tabular}
    }
  \end{tabularx}
}

\begin{document}
%-------------------------------------------------- BEGIN HERE --------------------------------------------------

%---------------------------------------------------- HEADER ----------------------------------------------------

\headertype{\email}{\github}{\linkedin}{\website}{}{} % Set the order of items here
\vspace{-30pt} % Set a negative value to push the body up, and the opposite

%-------------------------------------------------- BACKGROUND --------------------------------------------------
\section{\faImage}{Background Statement}

\begin{flushleft}
I have over a decade long experience in life sciences research. I have expertise in most modern experimental and analytical methods for high-throughput measurement, data processing, and systematic analysis. I also have research experience in protein chemistry, infectious microbiology, and cellular biology.
\end{flushleft}

%-------------------------------------------------- EDUCATION --------------------------------------------------
\section{\faGraduationCap}{Education}

  \resumeEntryStart
    \resumeEntryTSDL
      {Stanford University}{2015 -- now}
      {Ph.D. Bioengineering. Thesis advisers: Stephen Quake, John Boothroyd}{Stanford, CA, USA}
    \resumeEntryTSDL
      {Stanford University}{2015 -- 2017}
      {M.S. Bioengineering}{Stanford, CA, USA}
    \resumeEntryTSDL
      {UT Southwestern Medical Center}{2014 -- 2015}
      {Biophysics (enrolled in Ph.D. program)}{Dallas, TX, USA}
    \resumeEntryTSDL
      {Reed College}{2010 -- 2014}
      {B.A. Biology}{Portland, OR, USA}
  \resumeEntryEnd

%-------------------------------------------------- AWARDS --------------------------------------------------
\section{\faTrophy}{Awards \& Honors}

  \resumeEntryStart
    \resumeEntryTD
      {Stanford Bio-X Travel Award}{2019}
    \resumeEntryTD
      {Stanford Bio-X SIGF Fellow}{2018}
    \resumeItemListStart
      \resumeItem {One of 11 students awarded with a three-year fellowship to conduct interdisciplinary research on the topics of parasitology and single-cell bioinformatics co-advised by professors John Boothroyd and Stephen Quake}
    \resumeItemListEnd
    \resumeEntryTD
      {Reed College Larry Ruben Postbac. Research Fellow}{2014}
    \resumeEntryTD
      {Reed College Summer Experience Awardee}{2013}
    \resumeEntryTD
      {Reed College Independent Research Awardee}{2012}
    \resumeEntryTD
      {iGEM Competition Team Gold Medalist}{2009}
  \resumeEntryEnd

%-------------------------------------------------- PROJECTS --------------------------------------------------
\section{\faFlask}{Projects}
  \resumeEntryStart
    \resumeEntryTSDL
      {Life-cycle of Toxoplasma gondii and co-transcriptomic analysis of host infection}{2018 -- now}
      {\href{https://elifesciences.org/articles/54129}{published on eLife}}{}
    \resumeItemListStart
      \resumeItem {Produced the first single-cell atlas of Toxoplasma. Built an \href{http://st-atlas.org}{interactive visualization} website with Apache2, Flask, Bokeh. Comparative analysis to \textit{Plasmodium berghei} revealed concerted genetic program to life-cycle.}
    \resumeItemListEnd
  \resumeEntryEnd

  \resumeEntryStart
    \resumeEntryTSDL
      {A novel single-cell analysis algorithm: self-assembling manifolds (SAM)}{2017 -- 2019}
      {\href{https://elifesciences.org/articles/48994}{published on eLife}}{}
    \resumeItemListStart
      \resumeItem {Single-cell sequencing reveals novel germ cell population in a parasitic flatworm, Schistosoma mansoni.}
    \resumeItemListEnd
  \resumeEntryEnd

  \resumeEntryStart
    \resumeEntryTSDL
      {Temperature effect on DNA polymerase fidelity}{2015 -- 2020}
      {Submitted}{}
    \resumeItemListStart
      \resumeItem {Biochemically characterized error rate and activity of DNA polymerase of psychrophilic, mesophilic, and thermophilic origins.}
    \resumeItemListEnd
  \resumeEntryEnd


%-------------------------------------------------- PUBLICATIONS --------------------------------------------------
\newpage
\section{\faBook}{Publications}

\begin{enumerate}
    \item \textbf{Yuan Xue}, Ido Braslavsky, Stephen R. Quake. Temperature effect on DNA polymerase fidelity. \textbf{submitted (2020)}. \href{https://www.biorxiv.org/content/10.1101/2020.08.04.236919v1}{bioRxiv preprint}
    \item Pengyang Li, Dania Nanes Sarfati*, \textbf{Yuan Xue}*, Xi Yu, Alexander Tarashansky, Stephen R. Quake, Bo Wang. Single-cell analysis of Schistosoma mansoni reveals a conserved genetic program controlling germline stem cell fate. \textbf{accepted at Nature Communications (2020)}. \href{https://www.biorxiv.org/content/10.1101/2020.07.06.190033v1}{bioRxiv preprint}
    \item Suchita Rastogi, \textbf{Yuan Xue}, Stephen R. Quake*, John Boothroyd*. Differential Impacts on Host Transcription by ROP and GRA Effectors from the Intracellular Parasite Toxoplasma gondii. \textbf{mBio (2020)}. \href{https://mbio.asm.org/content/11/3/e00182-20}{article}, \href{https://www.biorxiv.org/content/10.1101/2020.02.04.934570v2.abstract}{bioRxiv preprint}
    \item \textbf{Yuan Xue}, Terence Theisen, Suchi Rastogi, Abel Ferrel, Stephen R. Quake*, John Boothroyd*. A single-parasite transcriptional atlas of Toxoplasma gondii reveals novel control of antigen expression. \textbf{eLife (2020)}. \href{https://elifesciences.org/articles/54129}{article}, \href{https://www.biorxiv.org/content/10.1101/656165v2}{bioRxiv preprint}
    \item Alexander Tarashansky, \textbf{Yuan Xue}, Pengyang Li, Stephen R. Quake, Bo Wang. Self-assembling Manifolds in Single-cell RNA Sequencing Data. \textbf{Elife (2019)}. \href{https://elifesciences.org/articles/48994}{article}, \href{https://www.biorxiv.org/content/10.1101/364166v2}{bioRxiv preprint}
    \item \textbf{The Tabula Muris Consortium}, Stephen R. Quake, Tony Wyss-Coray, Spyros Darmanis. Single-cell transcriptomics of 20 mouse organs creates a Tabula Muris. \textbf{Nature (2018)}. \href{https://www.nature.com/articles/s41586-018-0590-4}{article}, \href{https://www.biorxiv.org/content/10.1101/237446v2}{bioRxiv preprint}
    \item \textbf{Yuan Xue}, Jossef Osborn, Anand Panchal, Jay L. Mellies. The RpoE stress response pathway mediates reduction of enteropathogenic Escherichia coli virulence by zinc. \textbf{Applied and Environmental Microbiology (2015)}. \href{https://aem.asm.org/content/81/11/3766.short}{spotlight research article}
    \item  Jing Zhou, Shi-Hao Tan, Valerie Nicolas, Chantal Bauvy, Nai-Di Yang, Jianbin Zhang, \textbf{Yuan Xue}, Patrice Codogno, Han-Ming Shen. Activation of lysosomal function in the course of autophagy via mTORC1 suppression and autophagosome-lysosome fusion. \textbf{Cell Research (2013)}. \href{https://www.ncbi.nlm.nih.gov/pmc/articles/PMC3616426/}{article}

        *equal contributions
\end{enumerate}


%-------------------------------------------------- TEACHING --------------------------------------------------
\section{\faUsers}{Teaching Experience}

  \resumeEntryStart
    \resumeEntryTSDL
      {Teaching Assistant}{2018}
      {Microfluidic Device Laboratory (BioE301D)}{Stanford University}
    \resumeEntryTSDL
      {Teaching Assistant}{2017}
      {Introduction to Bioengineering (BioE80)}{Stanford University}
    \resumeEntryTSDL
      {Teaching Assistant}{2014}
      {Microbiology}{Reed College}
    \resumeEntryTSDL
      {Academic tutor}{2011 -- 2014}
      {Biology, Chemistry}{Reed College}
  \resumeEntryEnd

%-------------------------------------------------- POSTER --------------------------------------------------
\section{\faSlideshare}{Poster \& Conference}

  \resumeEntryStart
    \resumeEntryTSDL
      {Building a single-cell atlas of Toxoplasma interactome}{2019}
      {Invited talk}{National University of Singapore}
    \resumeEntryTSDL
      {Building a single-cell atlas of Toxoplasma interactome}{2019}
      {Invited talk}{Cell Symposia Single Cells: From Technology to Biology}
    \resumeEntryTSDL
      {Single-cell co-transcriptomic measurement resolves parasitic life cycle and host interactions}{2018}
      {Poster presentation}{Stanford Bioengineering department retreat}
    \resumeEntryTSDL
      {Building a single-cell atlas of Toxoplasma interactome}{2018}
      {Invited talk}{Stanford Microbiology \& Immunology department retreat}
    \resumeEntryTSDL
      {Cool biochemistry measured with a hot tool}{2017}
      {Poster presentation}{Stanford Bioengineering department retreat}
    \resumeEntryTSDL
      {Temperature adaptation and polymerase fidelity}{2017}
      {Poster presentation}{Gordon Research Conference (GRC): Nucleic Acids}

  \resumeEntryEnd


%-------------------------------------------------- SKILLS --------------------------------------------------
\section{\faBarChart}{Skills}
 \resumeEntryStart
  \resumeEntryS{Languages } {English, Cantonese, Mandarin, Japanese, Python, R, C++, Bash, \LaTeX}
  \resumeEntryS{Visualization } {matplotlib, plotly, seaborn, bokeh, networkX, graphviz, graph-tool}
  \resumeEntryS{Machine Learning } {numpy, pandas, sklearn, scikit-learn, tensorflow, keras}
  \resumeEntryS{Workflows } {snakemake, cloud computing (e.g. AWS, slurm)}
  \resumeEntryS{Web development } {Flask-REST backend, Apache2}
  \resumeEntryS{Bioinformatics } {STAR RNA aligner, htseq-count, salmon, velocyto, samtools, cell ranger, scanpy}
  \resumeEntryS{Experimental } {Smart-seq2, 10X single-cell, molecular biology techniques (e.g. cloning, expression), HPLC protein purification, FACS, tissue culturing, fluorescence imaging, viral transfection, high-throughput liquid-handling}
 \resumeEntryEnd

%-------------------------------------------------- PACKAGES --------------------------------------------------
\section{\faGithub}{Maintained Packages}
 \resumeEntryStart
  \resumeEntryS{\href{https://github.com/xuesoso/singleCell\_snake}{singleCell\_snake} } {A snakemake pipeline for local/Slurm submission of single cell data alignment andtranscript counting.}
  \resumeEntryS{\href{https://github.com/xuesoso/DensityPlot}{DensityPlot} } {A python package for generating density plot commonly seen in FACS analysis.}
  \resumeEntryS{\href{https://github.com/xuesoso/bag\_of\_velocyto}{bag\_of\_velocyto} } {A Slurm submission script for parallel submission of RNA velocity alignment.}
  \resumeEntryS{\href{https://github.com/xuesoso/nheatmap}{nheatmap} } {A multi-level heatmap plotting tool with the option to perform hierarchical clustering.}
 \resumeEntryEnd

\end{document}
